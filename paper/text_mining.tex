\documentclass[11pt,a4paper]{article}
\usepackage{acl2015}
\usepackage{url}
\usepackage{latexsym}

\title{Asignatura Text Mining II\\
M\'aster/Diploma Big Data Analytics}

\author{Pedro Henrique Mano Figueiredo Fernandes \\
  {\tt pedromorfeu@gmail.com} \\}

\date{\today}

\begin{document}
\maketitle
\begin{abstract}
  
  La aproximaci\'on al problema de Text Mining para Author Profiling ha tenido como base la t\'ecnica conocida por {\em bag of words}. Esta t\'ecnica emplea modelos para aprender el vocabul\'ario de un conjunto de documentos, a trav\'es del que se construye una representaci\'on de los datos en forma de matriz, adecuada para aplicar machine learning. A la matriz de vocabulario se han a\~nadido otras caracter\'isticas, como contadores de polaridad de sentimientos y determinadas estad\'isticas del texto de los documentos. El metadata de los tweets se ha usado también para reforzar las caracter\'isticas de la matriz.
  
  El modelo usado para el aprendizaje del vocabul\'ario se basa en contadores de palabras, calculando un coheficiente de tipo TF-IDF. Se ha configurado el modelo con un máximo de 2000 caracter\'isticas, que se traduce en el c\'alculo de las 2000 palabras más significativas en el corpus de entrenamiento. Para la divisi\'on de los documentos en trozos ({\em tokens}) se han aplicado {\em tokenizers} especializados en texto de Twitter.
  
  Una vez constru\'ida la matriz con todas las caracter\'isticas, se ha elegido un clasificador de tipo RandomForest para entrenar un modelo matem\'atico. Este clasificador es de tipo {\em ensemble}, aplicando varias iteraciones de predicci\'on sobre conjuntos aleat\'orios de los datos, lo que garantiza una mejor generalizaci\'on del modelo.

\end{abstract}


\section{Introducci\'on}

  La exploraci\'on de datos de lenguage natural permite descubrir características de los autores basadas en los patrones de escrita. El objectivo de este ejercicio es explorar la informaci\'on de un dataset constitu\'ido por tweets de varios usuarios de distintos pa\'ises de habla hisp\'anica, con el intuito de crear un modelo para inferir sus características (sexo y pa\'is).
  
   

  Breve introducci\'on al problema de Author Profiling y concretamente al presentado en clase. En este apartado el alumno deber\'a resumir en qu\'e consiste el problema y ponerlo en perspectiva para que el lector comprenda los siguientes apartados.


\section{Dataset}

Estad\'isticas del dataset que el alumno considere importantes. En clase se han visto las estad\'isticas b\'asicas del dataset y se ha explorado para obtener caracter\'isticas m\'as avanzadas. En este apartado el alumno tiene total libertad para exponer las tablas o gr\'aficas que considere apropiadas para describir el dataset, tanto desde un punto de vista ling¨u\'istico como de big data. 


\section{Propuesta del alumno}

Descripci\'on de la propuesta. Qu\'e caracter\'isticas se han utilizado y cu\'al ha sido la hip\'otesis para elegirlas. En clase se ha visto la construcci\'on de una baseline basada en bolsa de palabras. En este apartado el alumno expondr\'a las mejoras propuestas.

\section{Resultados experimentales}

Presentaci\'on de los resultados y an\'alisis de los mismos. La presentaci\'on de resultados y su an\'alisis implica mostrar en qu\'e contribuye la propuesta realizada, es decir, ¿son mejores los resultados?, ¿se procesan m\'as r\'apidos los datos?, ¿se aportan nuevas explicaciones conceptuales al problema?

\section{Conclusiones y trabajo futuro}

Breve presentaci\'on de las conclusiones sobre  el trabajo realizado e ideas de futuro para mejorar los resultados.


\begin{thebibliography}{}

\bibitem[\protect\citename{Aho and Ullman}1972]{Aho:72}
Alfred~V. Aho and Jeffrey~D. Ullman.
\newblock 1972.
\newblock {\em The Theory of Parsing, Translation and Compiling}, volume~1.
\newblock Prentice-{Hall}, Englewood Cliffs, NJ.

\end{thebibliography}

\end{document}
